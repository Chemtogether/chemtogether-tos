% !TEX program = xelatex

\documentclass[8pt,a4paper]{article}		

\usepackage{fontspec}

\usepackage{unicode-math}

\usepackage{xltxtra}

\usepackage[T1]{fontenc}

\usepackage{microtype}

\usepackage[utf8]{inputenc}

\usepackage{geometry}
\newgeometry{top=2cm, bottom=2cm, left=1.5cm, right=1.5cm, headheight=12pt}	

\usepackage[ngerman]{babel}
\selectlanguage{ngerman}

\usepackage{graphicx}

\usepackage{fancyhdr}

\usepackage{enumitem}

\usepackage{multicol}

\defaultfontfeatures{Ligatures=TeX,Numbers=OldStyle}
\setmainfont{Merriweather}
\setmathfont[Scale=MatchUppercase]{XITSMath-Regular.otf}
\setmathfont[range=up/{latin,Latin,num}]{Merriweather}
\setmathfont[range=it/{latin,Latin,num}]{Merriweather}
\setmathfont[range=bb/{latin,Latin,num}]{Merriweather}

													



\fancypagestyle{standardpage}{
	\fancyhf{}
	\lhead{}
	\chead{}
	\rhead{}
	\lfoot{}
	\cfoot{Seite \thepage}
	\rfoot{}
	\renewcommand{\headrulewidth}{0.0pt}
	\renewcommand{\footrulewidth}{0.0pt}}
\pagestyle{standardpage}


\setlength{\parindent}{0pt}		
\setlength{\parskip}{0pt}	
\setlength{\columnsep}{1cm}

\usepackage{titlesec}
\titleformat*{\section}{\large\bfseries}


\begin{document}

\pretolerance=5000
\tolerance=9000
\emergencystretch=0pt
\righthyphenmin=4
\lefthyphenmin=4


\renewcommand{\headrulewidth}{0pt}
\begin{minipage}[b]{0.5\textwidth}
	\LARGE{\textbf{Teilnahmebedingungen}}\par
	\large{für die Karrieremesse ``Chemtogether''}
\end{minipage}\hfill
\begin{minipage}[b]{0.4\textwidth}
	\includegraphics[width=\linewidth]{logo.png}
\end{minipage}

\vspace{2cm}

\begin{multicols}{2}

\section{Geltungsbereich}
Diese Teilnahmebedingungen gelten für Messeveranstaltungen des VSETH Verband der Studierenden an der ETH, Universitätsstrasse 6, 8092 Zürich, sowie seiner Fachvereine, welche nachfolgend gemeinsam als ``Veranstalter'' bezeichnet werden. Die Teilnahmebedingungen regeln die Rechtsbeziehungen zwischen dem Veranstalter und den Ausstellern, wobei allfällige individuelle Vereinbarungen zwischen den Parteien diesen Bedingungen vorgehen, sofern sie schriftlich (auch per E-Mail) vereinbart wurden.

\section{Vertragsgrundlagen}
Vertragsgrundlagen für die Teilnahme des Ausstellers an einer Messe des Veranstalters sind die Zulassungsbestätigung des Veranstalters, diese Teilnahmebedingungen sowie die Hausordnung der ETH Zürich. Ferner organisatorische, technische und sonstige Bestimmungen, die dem Aussteller vor Veranstaltungsbeginn zugestellt werden. Erbringt der Veranstalter aufgrund gesonderter Beauftragung weitere Messeservices durch Servicepartner, so gelten ergänzend die Allgemeinen Geschäftsbedingungen des jeweiligen Servicepartners.

\section{Anmeldung, Vertragsabschluss}
Die Anmeldung erfolgt ausschliesslich online über die Plattform der jeweiligen Messe, und die Korrespondenz erfolgt ausschliesslich über die auf der Plattform angegebene E-Mail-Adresse des Veranstalters. Mit der Anmeldung und entsprechender Aktivierung der Teilnahme an der Messe werden diese Teilnahmebedingungen vom anmeldenden Aussteller verbindlich anerkannt. Die Anmeldezeiten und -daten werden vom Veranstalter entsprechend erhoben und bis zum Eingang der kompletten Rechnungssumme gespeichert. Der Aussteller haftet dafür, dass auch sämtliche von ihm im Zusammenhang mit der Messe beschäftigten Personen diese Bedingungen einhalten. Mit der schriftlichen Zulassungsbestätigung durch den Veranstalter kommt der Mietvertrag zwischen Aussteller und Veranstalter zustande. Weicht der Inhalt der Zulassungsbestätigung vom Inhalt der Anmeldung ab, so kommt der Vertrag nach Massgabe der Zulassungsbestätigung zustande, es sei denn, der Aussteller widerspricht binnen zwei Wochen schriftlich.

\section{Zulassung}
Der Veranstalter entscheidet nach freiem Ermessen und auf Basis seiner Statuten über die Zulassung eines Ausstellers zur Messe. Der Aussteller nimmt zur Kenntnis, dass der ETH Zürich ein freies Mitspracherecht bei der Zulassung zustehen kann.

\section{Standflächenzuteilung}
Die Standflächenzuteilung erfolgt frei durch den Veranstalter. In der Anmeldung geäusserte Platzierungswünsche werden nach Möglichkeit beachtet. Der Veranstalter ist aber nötigenfalls auch nach Vertragsschluss jederzeit einseitig berechtigt, Grösse, Form, Kategorie und Lage der zugeteilten Standfläche zu verändern. Hierüber informiert er den Aussteller unverzüglich, wobei er ihm nach Möglichkeit eine gleichwertige andere Standfläche zuteilt. Verändert sich durch die Umteilung die Standmiete, so wird dem Aussteller eine angepasste neue Rechnung zugestellt. Der Aussteller akzeptiert mit seiner Anmeldung ausdrücklich und entschädigungslos, dass sich nach seiner Anmeldung die Lage, Grösse, Form, Kategorie und Verteilung der Standflächen noch verändert. Dem Aussteller stehen hieraus keinerlei Entschädigungsansprüche zu.

\section{Mitausteller}
Die Zulassung eines oder mehrerer Mitaussteller ist nur mit vorgängiger schriftlicher Zustimmung des Veranstalters zulässig.

\section{Preise}
Die Mietpreise für die beanspruchten Flächen (zonenabhängig) sind bei der Anmeldung auf der Online-Plattform der Messe mindestens als Richtpreise ausgewiesen und verbindlich. Bei der Berechnung wird die zugeteilte Bodenfläche ohne Rücksicht auf Vorsprünge, Pfeiler, Säulen, Installationsanschlüsse und sonstige feste Einbauten zugrunde gelegt. Es besteht kein Anspruch auf eine bestimmte Standplatzierung bzw. Zone. Im Mietpreis sind enthalten: 
\begin{itemize}
    \item Die mietweise Überlassung der Standfläche (inkl. Stromversorgung) während Aufbau, Laufzeit und Abbau Mietstand (falls in Kategorie enthalten); 
    \item die allgemeine Beleuchtung der Ausstellungshalle;
\end{itemize}
Allfällige Zusatzleistungen werden vom Veranstalter separat in Rechnung gestellt. Alle Preise verstehen sich exkl. gesetzlicher Mehrwertsteuer.

\section{Zahlungsbedingungen}
Die vom Veranstalter ausgestellten Rechnungen sind ohne Abzug zu den festgesetzten Terminen zahlbar. Sämtliche Zahlungen sind unter Angabe der Rechnungsnummer spesenfrei und in CHF zu entrichten. In jedem Fall muss der Rechnungsbetrag spätestens einen Tag vor der Messe auf dem Konto des Veranstalters gutgeschrieben sein. Andernfalls ist der Veranstalter berechtigt, dem Aussteller den Zutritt zum Stand zu verweigern. Der Aussteller hat sich nach der gewünschten Zahlungsmodalität des Veranstalters zu richten. 

\section{Rücktritt und Kündigung aus wichtigem Grund}
\subsection{Des Austellers}
Nach verbindlicher Anmeldung (Zustimmung zu diesen Teilnahmebedingungen) und Zulassung durch den Veranstalter ist ein Rücktritt des Ausstellers vom Vertrag nicht mehr möglich. Der Veranstalter kann dem Wunsch nach vorzeitiger Auflösung des Vertragsverhältnisses aber ausnahmsweise zustimmen, wenn die freiwerdende Standfläche anderweitig verwendet werden kann. In diesem Fall muss der vertragsauflösende Aussteller jedoch sämtliche aus dem Dahinfallen des Vertrages entstehenden Kosten übernehmen, insbesondere die Kosten für die Werbematerialien, allenfalls anfallende Mehrkosten sowie weitere Kosten infolge der Neuverwendung der Standfläche. Die vereinbarte Standmiete ist dennoch vollumfänglich vom Aussteller geschuldet, wenn der Veranstalter aus optischen oder logistischen Gründen die vom Aussteller nicht genutzte Fläche einem anderen Aussteller kostenlos oder zu einem reduzierten Tarif zuteilt. \\
Wird die dem Aussteller zugeteilte Standfläche um mehr als 20\% vergrössert oder verkleinert oder dem Aussteller eine andere Kategorie von Standplatz zugeteilt, so kann er innert 7 Tagen seit Zugang der entsprechenden Mitteilung des Veranstalters vom Vertrag zurücktreten. Ziff. 13 Abs. 2 gilt sinngemäss.

\subsection{Des Veranstalters}
Der Veranstalter ist befugt, nach erfolgter Mahnung und kurzer Nachfristansetzung vom Mietvertrag zurückzutreten oder diesen fristlos zu kündigen, wenn der Aussteller seinen Verpflichtungen aus dem Mietvertrag (insbesondere seinen Zahlungsverpflichtungen), diesen Teilnahmebedingungen oder den sie ergänzenden Bestimmungen nicht nachkommt. Entsprechendes gilt, wenn beim Aussteller die Voraussetzungen für den Vertragsabschluss nicht oder nicht mehr gegeben sind. 
Der Veranstalter ist in folgenden Fällen berechtigt, den Vertrag ohne Mahnung und Nachfristansetzung zu kündigen und die Standfläche anderweitig zu vergeben: wenn 
\begin{itemize}
    \item der Aussteller seine Zahlungen einstellt, über sein Vermögen das gerichtliche Nachlass- oder Konkursverfahren beantragt wird oder sich der Aussteller in Liquidation befindet. In diesen Fällen steht dem Veranstalter ein pauschaler Entschädigungsanspruch in der Höhe von 50\% der Netto-Grundmiete nebst Zuschlägen zu. Weiterer Schaden bleibt vorbehalten.
    \item die Standfläche nicht rechtzeitig, das heisst bis eine Stunde nach Veranstaltungsbeginn, erkennbar belegt ist.
    \item die Voraussetzungen für die Zulassung des angemeldeten Ausstellers nicht mehr gegeben sind oder dem Veranstalter nachträglich Gründe bekannt werden, deren rechtzeitige Kenntnis eine Nichtzulassung gerechtfertigt hätten.
    \item der Aussteller gegen die Hausordnung der ETH Zürich verstösst.
    \item die Ausstellungseinrichtung des Ausstellers gegen die Brandschutzvorschriften verstösst oder sonstige Sicherheitsvorschriften verletzt.
\end{itemize}
In diesen Fällen schuldet der Aussteller dem Veranstalter den gesamten ihm in Rechnung gestellten Betrag und haftet dem Veranstalter überdies für weiteren Schaden. 

\section{Standaufbau, Standausstattung, Standgestaltung}
Der Ausstellungsstand muss dem Thema und der Gestaltung der Messe angepasst sein. Der Veranstalter behält sich vor, den Aufbau unpassend oder unzureichend ausgestatteter Stände zu untersagen oder auf Kosten des Ausstellers abzuändern. Die Standfläche muss während der gesamten Veranstaltungsdauer zu den festgesetzten Öffnungszeiten ordnungsgemäss ausgestattet und mit fachkundigem Personal besetzt sein. Der Aufbau muss spätestens bis zum Aufbautermin abgeschlossen und der Stand von Verpackungsmaterial geräumt sein. Der Abtransport von Ausstellungsgütern und der Abbau von Ständen vor Schluss der Veranstaltung sind unzulässig.\\
Übersteigt der Messestand des Ausstellers die üblichen Raumhöhen (mehr als 2.30m), so bedarf der Stand der vorgängigen schriftlichen Zustimmung des Veranstalters. Das gleiche gilt für die Ausstellung von besonders schweren Ausstellungsgütern. Verankerungen im Hallenboden sind nicht zulässig. Das Anbringen von Plakaten jeglicher Art ausserhalb des Ausstellungsstandes ist ohne vorgängige schriftliche Zustimmung des Veranstalters unzulässig. \\
Der Veranstalter kann jederzeit vom Aussteller verlangen, dass Gegenstände entfernt werden, die sich als gefährdend, belästigend oder sonst wie ungeeignet erweisen. Wird dieser Aufforderung nicht in der festgesetzten Frist entsprochen, so erfolgt die Entfernung der Gegenstände ohne weitere Mahnung durch den Veranstalter und auf Kosten des Ausstellers.\\
Nach Beendigung der Messe ist die Stellfläche so wie sie vom Veranstalter bereitgestellt worden ist gereinigt und unbeschädigt zurückzugeben und der ursprüngliche Zustand wieder herzustellen. Wird der Messestand vom Aussteller dennoch nicht in angemessen gereinigtem Zustand hinterlassen, so ist der Veranstalter berechtigt, den Stand ohne Mahnung umgehend auf Kosten des Ausstellers reinigen zu lassen.\\
Schäden an der Infrastruktur des Veranstalters sind unverzüglich nach Schadenseintritt zu melden und werden auf Kosten des Ausstellers beseitigt. Ausstellungsgüter, die sich nach dem Abbauendtermin noch auf den Ständen befinden, können nach Belieben des Veranstalters auf Kosten des Ausstellers abtransportiert, entsorgt oder eingelagert werden.
Benötigt der Aussteller einen Transportdienst (Standbestandteile > 50kg) innerhalb der ETH-Räumlichkeiten, so muss der Veranstalter vorgängig schriftlich darüber informiert werden. Zudem hat der Aussteller zu gewährleisten, dass die anderen Aussteller durch diesen Transportdienst in ihrem Auf- und Abbau der Messestände nicht behindert werden.

\section{Haftung, Versicherung, Unfallschutz}
Der Veranstalter schliesst jegliche Haftungs- oder Regressansprüche bei Beschädigung, Verlust oder amtlicher Beschlagnahme von Standeinrichtungen, Ausstellungsgütern und fremden Gegenständen aus, sowohl für die Zeit, während derer sich die Güter am Messestandort befinden als auch während des Zu- und Abtransports (einschliesslich Lagerung bei Veranstalter). Die Haftung für Vorsatz und Grobe Fahrlässigkeit bleibt vorbehalten. Des Weiteren ist jede Haftung des Veranstalters ausgeschlossen für Schäden, die sich auf Grund von Vorführungen und Präsentationen von Ausstellern, durch den Auf- und Abbau von Ständen oder aus dem Standbetrieb ergeben. Der Aussteller ist für die Versicherung von Personen- und Sachschäden eigenverantwortlich. Ebenso ist ausschliesslich der Aussteller für die Versicherung sämtlicher Ausstellungsgüter und Standeinrichtungen während der Messe und während des Zu- und Abtransports (Einschliesslich Lagerung beim Veranstalter) gegen Beschädigung und Verlust sowie für den Abschluss einer Haftpflichtverssicherung verantwortlich. Der Aussteller ist verpflichtet, an den ausgestellten Maschinen und Geräten Schutzvorrichtungen anzubringen, die den gesetzlichen Unfallverhütungsvorschriften entsprechen. Der Veranstalter ist berechtigt, das Ausstellen oder die Inbetriebnahme von Maschinen und Geräten nach seinem Ermessen zu untersagen.

\section{Brandschutz}
Ausstellungseinrichtungen (exkl. Exponate) müssen mindestens Brandkennziffer BKZ 6q3 aufweisen oder in der Brandverhaltensgruppe RF1 sein. Tischtücher, Plakate und weiteres Material, welches vom Aussteller zu Dekorationszwecken verwendet wird, müssen entweder der Brandverhaltensgruppe RF2 angehören oder mindestens Brandkennziffer BKZ 5.2 aufweisen.

\section{Absage, Abbruch, Verschiebung oder Anpassung einer Messe}
Der Veranstalter behält sich vor, die Veranstaltung abzusagen, örtlich und zeitlich zu verlegen oder die Dauer zu verändern. Hieraus ergibt sich für den Aussteller kein Recht, vom Vertrag zurückzutreten. Abs. 4 bleibt vorbehalten. Kann der Veranstalter auf Grund höherer Gewalt oder politischer Ereignisse die Veranstaltung nicht durchführen, so hat er die Aussteller hiervon unverzüglich zu unterrichten.\\
Im Falle einer Absage oder zeitlichen Verschiebung der Messe entfällt der Anspruch auf Standmiete, jedoch kann der Veranstalter ihm vom Aussteller in Auftrag gegebene Arbeiten im Umfang der entstandenen Aufwendungen in Rechnung stellen, so weit das Ergebnis der Arbeiten für den Aussteller noch von Interesse ist.\\
Sollte der Veranstalter in der Lage sein, die Veranstaltung zu einem späteren Termin durchzuführen, so hat er die Aussteller hiervon unverzüglich zu unterrichten. Die Aussteller sind berechtigt, innerhalb einer Woche nach Zugang dieser Mitteilung ihre Teilnahme zu dem veränderten Termin abzusagen. In diesem Falle haben sie Anspruch auf Rückerstattung bzw. Erlass der Standmiete.\\
Wird die Messe örtlich verschoben oder aufgrund höherer Gewalt oder politischer Ereignisse abgesagt oder muss eine begonnene Veranstaltung aufgrund höherer Gewalt oder politischer Ereignisse abgebrochen werden, so hat der Aussteller keinen Anspruch auf Rückzahlung oder Erlass der Standmiete oder Ersatz seiner Aufwendungen oder sonstigen Schadens.

\section{Hausordnung, Zuwiederhandlungen}
Der Aussteller verpflichtet sich, während der Messe auf dem gesamten Veranstaltungsgelände die Hausordnung der ETH Zürich einzuhalten. Weisungen des Veranstalters, des ETH-Personals und der Ordnungskräfte sind Folge zu leisten. Wer deren Anweisungen nicht befolgt, kann nach fruchtloser Verwarnung vom Veranstaltungsgelände weggewiesen werden, ohne dass ihm dadurch irgendwelche Rechtsansprüche entstehen.\\
Aussteller, die den Vorschriften, Bedingungen und Reglementen zuwider handeln oder deren Verhalten an der Messe Anlass zu Beanstandungen gibt, können vom Veranstalter jederzeit entschädigungslos von der Messe ausgeschlossen werden. Der Aussteller haftet diesfalls dem Veranstalter in vollem Umfang für die Kosten und allfällige Folgekosten.

\section{Werbung, Fotografieren, Befragungen}
Werbung aller Art ist nur innerhalb der vom Aussteller gemieteten Standfläche und für die Firma des Ausstellers erlaubt. Die Verwendung von Geräten und Einrichtungen, durch die auf optische oder akustische Weise eine gesteigerte Werbewirkung erzielt werden soll, bedürfen der vorgängigen schriftlichen Zustimmung des Veranstalters. Werbung politischen Charakters ist grundsätzlich unzulässig. Zudem ist der Aussteller nicht berechtigt, ohne vorgängige schriftliche Zustimmung des Veranstalters ausserhalb des Messegeländes mit seiner Teilnahme an der Messe Werbung zu betreiben.\\
Der Veranstalter ist berechtigt, Fotografien, Zeichnungen, Ton- und Filmaufnahmen vom Messegeschehen, von den Ausstellungsbauten und -ständen und den ausgestellten Gegenständen anfertigen zu lassen und für Werbezwecke oder Presseveröffentlichungen zu verwenden, ohne dass der Aussteller aus irgendwelchen Gründen Einwendungen dagegen erheben kann. Das gilt auch für Aufnahmen, die Presse oder Fernsehen mit Zustimmung des Veranstalters direkt anfertigen. Befragungen seitens der Aussteller sind nur auf der eigenen Standfläche zulässig.

\section{Ausstelleransprüche, Schriftform}
Alle Ansprüche des Ausstellers gegen den Veranstalter sind schriftlich geltend zu machen. Die Verjährungsfrist beginnt mit dem ersten Tag der Messeveranstaltung. Vereinbarungen, die von diesen Bedingungen oder den sie ergänzenden Bestimmungen abweichen, bedürfen der Schriftform (auch E-Mail). 

\section{Gerichtsstand und anwendbares Recht}
Sämtliche Streitigkeiten aus oder in Zusammenhang mit diesen Teilnahmebedingungen unterliegen der \textbf{Gerichtsbarkeit der Gerichte am Sitz des Veranstalters}. Erfüllungsort ist ebenfalls am Sitz des Veranstalters. Der Veranstalter ist jedoch nach seiner freien Wahl berechtigt, seine Ansprüche am Sitz des Ausstellers geltend zu machen. 

\vspace{.75cm}
\textbf{Es ist ausschliesslich Schweizer Recht anwendbar, unter Ausschluss des Wiener Kaufrechts.}

\vspace{.75cm}
Zürich, 30.11.2019

\end{multicols}
\end{document}